\documentclass[12pt,a4paper,violet]{bbe}
\begin{document}
    	\chapter{Agile Software Development}
	\section{The Fundamentals of Agile Software Development}
	\begin{itemize}
	    \item The members in an Agile project communicate with each other early and frequently which helps with removing defects early and developing a quality product.
	\end{itemize}
	\subsection{Agile Software Development and Agile Manifesto}
	\begin{itemize}
	    \item In 2001, a group of individuals agreed on a common set of values and principles which became known as Agile Manifesto.
	\end{itemize}
		    \begin{remark}
		    \textbf{Four Statement of Values} \\
	           Agile Manifesto contains four statement of values
	            	\begin{itemize}
	                    \item Individuals and interactions over processes and tools
	                    \item Working software over comprehensive documentation
	                    \item Customer collaboration over contract negotiation
	                    \item Responding to change over following a plan
            	    \end{itemize}         
	        \end{remark}
	  \textbf{Individuals and Interactions} \\
	  Agile is people centered. Continuous communication and interaction is more important than reliance on tools and processes.\\
	  \textbf{Working Software} \\
	  Working software is more useful and valuable than detailed documentation. It provides opportunity to give the development team rapid feedback. \\
	  Agile development especially useful in rapidly changing business environments where the problems and solutions are unclear or where the business wishes to innovate in new problem domains.\\
	  \textbf{Customer Collaboration} \\
	  Customers often find great difficulty in specifying the system they require. Collaborating directly with the customer  improves the likelihood of understanding exactly what customer requires. \\	 
	  \textbf{Responding to Change} \\
	  Change is inevitable in software projects. Having flexibility in work practices to embrace change is more important than simply adhering rigidly to a plan \\	
	  \begin{remark}
	      \textbf{Agile Manifesto Principles} \\
	      \begin{itemize}
	          \item Highest priority is satisfy the customer with early and continuous delivery of value.
	          \item Welcoming changing requirements, even in late development.
	          \item Deliver working software frequently between a few weeks to a few months.
	          \item Business people and developers must work together daily throughout the project.
	          \item Build project around motivated individuals, give them the environment and support they need and trust them to get the job done.
	          \item The most efficient and effective method of conveying information is face to face conversation with the development team.
	          \item Working software is the primary measure of process.
	          \item Agile processes promote sustainable development. The sponsors, developers and users should be able to maintain a constant pace indefinitely.
	          \item Continuous attention to technical excellence  and good design enhances agility.
	          \item Simplicity is the art of maximizing the amount of work not done is essential.
	          \item The best architectures, requirements and design emerge from self organized teams.
	          \item At regular intervals, the team reflects on how to become more effective, then tunes and adjusts its behavior accordingly.
	      \end{itemize}
	  \end{remark}
	
	\begin{definition}
        The agile community has a saying: \textbf{“Simplicity is the art of maximizing the work not done.”} This idea is central to eliminating waste. To make your process more agile, do less. However, you can't take so much out of your process that it no longer works.
	\end{definition}
		\begin{definition}
        \textbf{Self organized team} is one that does not depend on or wait for a manager to assign work. Instead, these teams find their own work and manage the associated responsibilities and timelines.
	\end{definition}
	
    \subsection{Whole-Team Approach}
    The whole-team approach means involving everyone with the knowledge and skills necessary to ensure project success. The team includes representatives from the customer and other business stakeholders who determine product features. The essence of the whole-team approach lies in the testers, developers and the business representatives working together in every step of the development process. 
    \begin{remark}
        The whole team is involved in any consulations or meetings which product features are  presented, analyzed or estimated. The concept of involving testes, developers and business representatives in all feature  discussions is known as \textbf{the power of three}.
    \end{remark}
    \begin{itemize}
        \item The team should be relatively small. (3-9 people)
        \item Ideally, the whole team shares the same workspace as co-location strongly facilitates communication and interactions
        \item The whole team approach is supported through the daily stand-up meetings involving all members of the team where work progress is communicated and any impediments to progress are highlighted.
        \item The whole-team approach promotes more effective and efficient team dynamics.
    \end{itemize}
    
    \begin{remark}
        Whole-team approach is one of the main benefits of Agile development. Its benefits include:
        \begin{itemize}
            \item Enhancing communication and collaboration within the team.
            \item Enabling various skill sets within the team to be leveraged to the benefit of the project.
            \item Making quality everyone's responsibility.
        \end{itemize}
    \end{remark}
    \subsection{Early and Frequent Feedback}
    Agile projects have short iterations enabling the project team to receive early and continuous feedback on product quality throughout the development lifecycle. One way to provide rapid feedback is by \textbf{continuous integration}. \\
    When sequential development approaches are used, the customer often does not see the product until the project is nearly completed. At that point , it is often too late for the development team to effectively address any issues the customer may have. By getting frequent feedback , teams can incorporate most new changes into the product development process. \\
    Early and frequent feedback helps the team to focus on the features with the highest business value, or associated risk, and these are delivered to the customer first.\\
    It also helps manage the team better since the capability of the team is transparent to everyone.
    \begin{remark}
    Benefits of early and frequent feedback include:
    \begin{itemize}
        \item Avoiding requirements misunderstandings, which may not have been detected until late in the development cycle when they are more expensive to fix.
        \item Clarifying customer feature requests, making them available for customer use early. Product better reflects what the customer wants.
        \item Discovering, isolating and resolving quality problems early.
        \item Promoting consistent project momentum.
    \end{itemize}
    \end{remark}
\section{Aspects of Agile Approaches}    
There are a number of agile approaches in  use by organizations. Common practices across most agile organizations include collaborative user story creation, retrospectives, continuous integration and planning for each iteration as well as for overall release.
\subsection{Agile Software Development Approaches}
There are several agile approaches, each of which implements the values and principles of agile manifesto in different ways. Some approaches are: Extreme Programming(XP) , Scrum, Kanban.
\subsubsection{Extreme Programming}
It is introduced by Kent Beck is an agile approach to software development described by certain values, principles and development practices. Many of the agile approaches in use today are influenced by XP and its values and princeples. \\
\begin{itemize}
    \item \textbf{XP embraces five values to guide development:} communication, simplicity, feedback, courage and respect.
    \item \textbf{XP describes  a set of principles as additional guidelines:} humanity, economics, mutual benefit,  self-similarity, improvement, diversity, reflection, flow, opportunity, redundancy, failure, quality, baby steps and accepted responsibility.
    \item \textbf{XP describes thirteen primary practices:} sit together, whole team, informative workspace, energized work, pair programming, stories, weekly cycle, quarterly cycle, stack, ten-minute build, continuous integration, test first programming and incremental design.
\end{itemize}
\subsubsection{Scrum}
Scrum is an agile management framework which contains the following practices:
\begin{remark}
    \begin{itemize}
    \item \textbf{Sprint:}  Scrum divides a project into iterations called sprints of fixed length (usually two to four weeks)
    \item \textbf{Product Increment:}  Each sprint results in a potentially releasable product called an increment 
    \item \textbf{Product Backlog:}  The product owner manages a prioritized list of planned product items called product backlog . The product backlog evolves from sprint to sprint called backlog refinement
    \item \textbf{Definition of Done:}  To make sure that there is a potentially releasable product at each sprint's end.  The scrum team discusses and defines appropriate criteria for sprint completion. 
    \item \textbf{Timeboxing:} If a development team cannot finish a task within the sprint, task is moved back into product backlog. Timeboxing applies not only tasks but also in meeeting start and end times.
    \item \textbf{Transparency:}  The development team reports and updates sprint status on daily basis at a meeting calles daily scrum. This makes the content and progress of the current sprint, including test results, visible to team, management and all interested parties.
    \end{itemize}
    \end{remark}

Scrum defines three roles: Scrum Master, Product Owner, Development Team
	\begin{definition}
        \textbf{Scrum Master} Ensures that Scrum practices and rules are implemented and followed, and resolved any violations, resource issues, or other impediments that could prevent the team from following the practices and rules. This person is not a team lead but a coach.
	\end{definition}
	
	\begin{definition}
        \textbf{Product Owner} Represents the customer and generates, maintains and prioritizes the product backlog. This person is not the team lead.
	\end{definition}
	
	\begin{definition}
        \textbf{Development Team} Develop and test the product. The team is self-organized: There is no team lead, so team makes the decision. Team is also cross-functional.
	\end{definition}
	
	\begin{remark}
    \begin{itemize}
    \item Scrum does not dictate specific software development techniques unlike XP. In addition Scrum doesn't provide guidance on how testing has to be done in a Scrum project.
    \end{itemize}
    \end{remark}
    
    \subsubsection{Kanban}
    Kanban is a management approach that is sometimes used in Agile projects. General objective is to visualize and optimize the flow of work within a value added chain. Kanban utilizes three instruments.
    \begin{remark}
    \begin{itemize}
    \item Kanban Board: Each column shows a station which is a set of related activities. Processes are symbolized by tickets moving from left to right.
    \item Work-in Progress Limit: The amount of parallel active tasks is strictly limited. This is controlled by the max number of tickets allowed for station or globally for the board. Whenever a station has free capacity, the worker pull a ticket from the predecessor station.
    \item Lead Time: Kanban is used to optimize the continuous flow of tasks by minimizing the average lead time for complete value stream.
    \end{itemize}
    \end{remark}
    
    Tasks not yet scheduled are waiting in the backlog and moved onto the Kanban board as soon as there is new space available.
    
	Iterations or sprints are optional in Kanban. The Kanban allows releasing its deliverables item by item, rather than as part of a release.
	
	\subsection{Colloborative User Story Creation}
\chapter{new}

\end{document}